% Options for packages loaded elsewhere
\PassOptionsToPackage{unicode}{hyperref}
\PassOptionsToPackage{hyphens}{url}
\PassOptionsToPackage{dvipsnames,svgnames,x11names}{xcolor}
%
\documentclass[
  letterpaper,
  DIV=11,
  numbers=noendperiod]{scrartcl}

\usepackage{amsmath,amssymb}
\usepackage{iftex}
\ifPDFTeX
  \usepackage[T1]{fontenc}
  \usepackage[utf8]{inputenc}
  \usepackage{textcomp} % provide euro and other symbols
\else % if luatex or xetex
  \usepackage{unicode-math}
  \defaultfontfeatures{Scale=MatchLowercase}
  \defaultfontfeatures[\rmfamily]{Ligatures=TeX,Scale=1}
\fi
\usepackage{lmodern}
\ifPDFTeX\else  
    % xetex/luatex font selection
\fi
% Use upquote if available, for straight quotes in verbatim environments
\IfFileExists{upquote.sty}{\usepackage{upquote}}{}
\IfFileExists{microtype.sty}{% use microtype if available
  \usepackage[]{microtype}
  \UseMicrotypeSet[protrusion]{basicmath} % disable protrusion for tt fonts
}{}
\makeatletter
\@ifundefined{KOMAClassName}{% if non-KOMA class
  \IfFileExists{parskip.sty}{%
    \usepackage{parskip}
  }{% else
    \setlength{\parindent}{0pt}
    \setlength{\parskip}{6pt plus 2pt minus 1pt}}
}{% if KOMA class
  \KOMAoptions{parskip=half}}
\makeatother
\usepackage{xcolor}
\setlength{\emergencystretch}{3em} % prevent overfull lines
\setcounter{secnumdepth}{-\maxdimen} % remove section numbering
% Make \paragraph and \subparagraph free-standing
\makeatletter
\ifx\paragraph\undefined\else
  \let\oldparagraph\paragraph
  \renewcommand{\paragraph}{
    \@ifstar
      \xxxParagraphStar
      \xxxParagraphNoStar
  }
  \newcommand{\xxxParagraphStar}[1]{\oldparagraph*{#1}\mbox{}}
  \newcommand{\xxxParagraphNoStar}[1]{\oldparagraph{#1}\mbox{}}
\fi
\ifx\subparagraph\undefined\else
  \let\oldsubparagraph\subparagraph
  \renewcommand{\subparagraph}{
    \@ifstar
      \xxxSubParagraphStar
      \xxxSubParagraphNoStar
  }
  \newcommand{\xxxSubParagraphStar}[1]{\oldsubparagraph*{#1}\mbox{}}
  \newcommand{\xxxSubParagraphNoStar}[1]{\oldsubparagraph{#1}\mbox{}}
\fi
\makeatother

\usepackage{color}
\usepackage{fancyvrb}
\newcommand{\VerbBar}{|}
\newcommand{\VERB}{\Verb[commandchars=\\\{\}]}
\DefineVerbatimEnvironment{Highlighting}{Verbatim}{commandchars=\\\{\}}
% Add ',fontsize=\small' for more characters per line
\usepackage{framed}
\definecolor{shadecolor}{RGB}{241,243,245}
\newenvironment{Shaded}{\begin{snugshade}}{\end{snugshade}}
\newcommand{\AlertTok}[1]{\textcolor[rgb]{0.68,0.00,0.00}{#1}}
\newcommand{\AnnotationTok}[1]{\textcolor[rgb]{0.37,0.37,0.37}{#1}}
\newcommand{\AttributeTok}[1]{\textcolor[rgb]{0.40,0.45,0.13}{#1}}
\newcommand{\BaseNTok}[1]{\textcolor[rgb]{0.68,0.00,0.00}{#1}}
\newcommand{\BuiltInTok}[1]{\textcolor[rgb]{0.00,0.23,0.31}{#1}}
\newcommand{\CharTok}[1]{\textcolor[rgb]{0.13,0.47,0.30}{#1}}
\newcommand{\CommentTok}[1]{\textcolor[rgb]{0.37,0.37,0.37}{#1}}
\newcommand{\CommentVarTok}[1]{\textcolor[rgb]{0.37,0.37,0.37}{\textit{#1}}}
\newcommand{\ConstantTok}[1]{\textcolor[rgb]{0.56,0.35,0.01}{#1}}
\newcommand{\ControlFlowTok}[1]{\textcolor[rgb]{0.00,0.23,0.31}{\textbf{#1}}}
\newcommand{\DataTypeTok}[1]{\textcolor[rgb]{0.68,0.00,0.00}{#1}}
\newcommand{\DecValTok}[1]{\textcolor[rgb]{0.68,0.00,0.00}{#1}}
\newcommand{\DocumentationTok}[1]{\textcolor[rgb]{0.37,0.37,0.37}{\textit{#1}}}
\newcommand{\ErrorTok}[1]{\textcolor[rgb]{0.68,0.00,0.00}{#1}}
\newcommand{\ExtensionTok}[1]{\textcolor[rgb]{0.00,0.23,0.31}{#1}}
\newcommand{\FloatTok}[1]{\textcolor[rgb]{0.68,0.00,0.00}{#1}}
\newcommand{\FunctionTok}[1]{\textcolor[rgb]{0.28,0.35,0.67}{#1}}
\newcommand{\ImportTok}[1]{\textcolor[rgb]{0.00,0.46,0.62}{#1}}
\newcommand{\InformationTok}[1]{\textcolor[rgb]{0.37,0.37,0.37}{#1}}
\newcommand{\KeywordTok}[1]{\textcolor[rgb]{0.00,0.23,0.31}{\textbf{#1}}}
\newcommand{\NormalTok}[1]{\textcolor[rgb]{0.00,0.23,0.31}{#1}}
\newcommand{\OperatorTok}[1]{\textcolor[rgb]{0.37,0.37,0.37}{#1}}
\newcommand{\OtherTok}[1]{\textcolor[rgb]{0.00,0.23,0.31}{#1}}
\newcommand{\PreprocessorTok}[1]{\textcolor[rgb]{0.68,0.00,0.00}{#1}}
\newcommand{\RegionMarkerTok}[1]{\textcolor[rgb]{0.00,0.23,0.31}{#1}}
\newcommand{\SpecialCharTok}[1]{\textcolor[rgb]{0.37,0.37,0.37}{#1}}
\newcommand{\SpecialStringTok}[1]{\textcolor[rgb]{0.13,0.47,0.30}{#1}}
\newcommand{\StringTok}[1]{\textcolor[rgb]{0.13,0.47,0.30}{#1}}
\newcommand{\VariableTok}[1]{\textcolor[rgb]{0.07,0.07,0.07}{#1}}
\newcommand{\VerbatimStringTok}[1]{\textcolor[rgb]{0.13,0.47,0.30}{#1}}
\newcommand{\WarningTok}[1]{\textcolor[rgb]{0.37,0.37,0.37}{\textit{#1}}}

\providecommand{\tightlist}{%
  \setlength{\itemsep}{0pt}\setlength{\parskip}{0pt}}\usepackage{longtable,booktabs,array}
\usepackage{calc} % for calculating minipage widths
% Correct order of tables after \paragraph or \subparagraph
\usepackage{etoolbox}
\makeatletter
\patchcmd\longtable{\par}{\if@noskipsec\mbox{}\fi\par}{}{}
\makeatother
% Allow footnotes in longtable head/foot
\IfFileExists{footnotehyper.sty}{\usepackage{footnotehyper}}{\usepackage{footnote}}
\makesavenoteenv{longtable}
\usepackage{graphicx}
\makeatletter
\def\maxwidth{\ifdim\Gin@nat@width>\linewidth\linewidth\else\Gin@nat@width\fi}
\def\maxheight{\ifdim\Gin@nat@height>\textheight\textheight\else\Gin@nat@height\fi}
\makeatother
% Scale images if necessary, so that they will not overflow the page
% margins by default, and it is still possible to overwrite the defaults
% using explicit options in \includegraphics[width, height, ...]{}
\setkeys{Gin}{width=\maxwidth,height=\maxheight,keepaspectratio}
% Set default figure placement to htbp
\makeatletter
\def\fps@figure{htbp}
\makeatother

\KOMAoption{captions}{tableheading}
\makeatletter
\@ifpackageloaded{tcolorbox}{}{\usepackage[skins,breakable]{tcolorbox}}
\@ifpackageloaded{fontawesome5}{}{\usepackage{fontawesome5}}
\definecolor{quarto-callout-color}{HTML}{909090}
\definecolor{quarto-callout-note-color}{HTML}{0758E5}
\definecolor{quarto-callout-important-color}{HTML}{CC1914}
\definecolor{quarto-callout-warning-color}{HTML}{EB9113}
\definecolor{quarto-callout-tip-color}{HTML}{00A047}
\definecolor{quarto-callout-caution-color}{HTML}{FC5300}
\definecolor{quarto-callout-color-frame}{HTML}{acacac}
\definecolor{quarto-callout-note-color-frame}{HTML}{4582ec}
\definecolor{quarto-callout-important-color-frame}{HTML}{d9534f}
\definecolor{quarto-callout-warning-color-frame}{HTML}{f0ad4e}
\definecolor{quarto-callout-tip-color-frame}{HTML}{02b875}
\definecolor{quarto-callout-caution-color-frame}{HTML}{fd7e14}
\makeatother
\makeatletter
\@ifpackageloaded{caption}{}{\usepackage{caption}}
\AtBeginDocument{%
\ifdefined\contentsname
  \renewcommand*\contentsname{Table of contents}
\else
  \newcommand\contentsname{Table of contents}
\fi
\ifdefined\listfigurename
  \renewcommand*\listfigurename{List of Figures}
\else
  \newcommand\listfigurename{List of Figures}
\fi
\ifdefined\listtablename
  \renewcommand*\listtablename{List of Tables}
\else
  \newcommand\listtablename{List of Tables}
\fi
\ifdefined\figurename
  \renewcommand*\figurename{Figure}
\else
  \newcommand\figurename{Figure}
\fi
\ifdefined\tablename
  \renewcommand*\tablename{Table}
\else
  \newcommand\tablename{Table}
\fi
}
\@ifpackageloaded{float}{}{\usepackage{float}}
\floatstyle{ruled}
\@ifundefined{c@chapter}{\newfloat{codelisting}{h}{lop}}{\newfloat{codelisting}{h}{lop}[chapter]}
\floatname{codelisting}{Listing}
\newcommand*\listoflistings{\listof{codelisting}{List of Listings}}
\makeatother
\makeatletter
\makeatother
\makeatletter
\@ifpackageloaded{caption}{}{\usepackage{caption}}
\@ifpackageloaded{subcaption}{}{\usepackage{subcaption}}
\makeatother

\ifLuaTeX
  \usepackage{selnolig}  % disable illegal ligatures
\fi
\usepackage{bookmark}

\IfFileExists{xurl.sty}{\usepackage{xurl}}{} % add URL line breaks if available
\urlstyle{same} % disable monospaced font for URLs
\hypersetup{
  pdftitle={SEM-Based Univariate Meta-Analysis},
  colorlinks=true,
  linkcolor={blue},
  filecolor={Maroon},
  citecolor={Blue},
  urlcolor={Blue},
  pdfcreator={LaTeX via pandoc}}


\title{SEM-Based Univariate Meta-Analysis}
\author{}
\date{}

\begin{document}
\maketitle


\begin{verbatim}
Loading required package: OpenMx
\end{verbatim}

\begin{verbatim}
OpenMx may run faster if it is compiled to take advantage of multiple cores.
\end{verbatim}

\begin{verbatim}
"SLSQP" is set as the default optimizer in OpenMx.
\end{verbatim}

\begin{verbatim}
mxOption(NULL, "Gradient algorithm") is set at "central".
\end{verbatim}

\begin{verbatim}
mxOption(NULL, "Optimality tolerance") is set at "6.3e-14".
\end{verbatim}

\begin{verbatim}
mxOption(NULL, "Gradient iterations") is set at "2".
\end{verbatim}

\section{Structural Equation Modeling (SEM) Based Univariate
Meta-Analysis}\label{structural-equation-modeling-sem-based-univariate-meta-analysis}

\begin{center}\rule{0.5\linewidth}{0.5pt}\end{center}

\subsection*{\texorpdfstring{1.
\textbf{Introduction}}{1. Introduction}}\label{introduction}
\addcontentsline{toc}{subsection}{1. \textbf{Introduction}}

\begin{tcolorbox}[enhanced jigsaw, toptitle=1mm, colframe=quarto-callout-note-color-frame, left=2mm, colbacktitle=quarto-callout-note-color!10!white, opacitybacktitle=0.6, bottomtitle=1mm, bottomrule=.15mm, coltitle=black, titlerule=0mm, title=\textcolor{quarto-callout-note-color}{\faInfo}\hspace{0.5em}{Key Conceptual Foundations}, opacityback=0, toprule=.15mm, rightrule=.15mm, colback=white, breakable, arc=.35mm, leftrule=.75mm]

\begin{itemize}
\tightlist
\item
  \textbf{SEM-Meta Integration}:\\
  Treat studies as ``subjects'' in SEM frameworks, where:

  \begin{itemize}
  \tightlist
  \item
    \textbf{Observed variables} = Reported effect sizes (e.g., SMD, odds
    ratios).\\
  \item
    \textbf{Latent variables} = True population effects (modeled as
    unobserved constructs).\\
  \end{itemize}
\item
  \textbf{Advantages}:

  \begin{itemize}
  \tightlist
  \item
    Handle missing data via \textbf{Full Information Maximum Likelihood
    (FIML)}.\\
  \item
    Fix known sampling variances using \textbf{definition variables}.\\
  \item
    Visualize models with path diagrams (e.g., latent heterogeneity).
  \end{itemize}
\end{itemize}

\end{tcolorbox}

\begin{center}\rule{0.5\linewidth}{0.5pt}\end{center}

\subsection{\texorpdfstring{2. \textbf{Computing Effect
Sizes}}{2. Computing Effect Sizes}}\label{computing-effect-sizes}

\subsubsection{Standardized Mean Difference
(SMD)}\label{standardized-mean-difference-smd}

\textbf{Equations}: \[
y_{\text{SMD}} = \frac{\bar{X}_1 - \bar{X}_2}{S_{\text{pooled}}}, \quad 
S_{\text{pooled}} = \sqrt{\frac{(n_1-1)S_1^2 + (n_2-1)S_2^2}{n_1 + n_2 - 2}}
\]

\textbf{Sampling Variance}: \[
v_{\text{SMD}} = \frac{n_1 + n_2}{n_1 n_2} + \frac{y_{\text{SMD}}^2}{2(n_1 + n_2)}
\]

\textbf{R Code}:

\begin{Shaded}
\begin{Highlighting}[]
\NormalTok{compute\_SMD }\OtherTok{\textless{}{-}} \ControlFlowTok{function}\NormalTok{(m1, m2, sd1, sd2, n1, n2) \{}
\NormalTok{  pooled\_sd }\OtherTok{\textless{}{-}} \FunctionTok{sqrt}\NormalTok{(((n1 }\SpecialCharTok{{-}} \DecValTok{1}\NormalTok{)}\SpecialCharTok{*}\NormalTok{sd1}\SpecialCharTok{\^{}}\DecValTok{2} \SpecialCharTok{+}\NormalTok{ (n2 }\SpecialCharTok{{-}} \DecValTok{1}\NormalTok{)}\SpecialCharTok{*}\NormalTok{sd2}\SpecialCharTok{\^{}}\DecValTok{2}\NormalTok{) }\SpecialCharTok{/}\NormalTok{ (n1 }\SpecialCharTok{+}\NormalTok{ n2 }\SpecialCharTok{{-}} \DecValTok{2}\NormalTok{))}
\NormalTok{  smd }\OtherTok{\textless{}{-}}\NormalTok{ (m1 }\SpecialCharTok{{-}}\NormalTok{ m2) }\SpecialCharTok{/}\NormalTok{ pooled\_sd}
\NormalTok{  v\_smd }\OtherTok{\textless{}{-}}\NormalTok{ (n1 }\SpecialCharTok{+}\NormalTok{ n2)}\SpecialCharTok{/}\NormalTok{(n1 }\SpecialCharTok{*}\NormalTok{ n2) }\SpecialCharTok{+}\NormalTok{ smd}\SpecialCharTok{\^{}}\DecValTok{2}\SpecialCharTok{/}\NormalTok{(}\DecValTok{2}\SpecialCharTok{*}\NormalTok{(n1 }\SpecialCharTok{+}\NormalTok{ n2))}
  \FunctionTok{return}\NormalTok{(}\FunctionTok{data.frame}\NormalTok{(}\AttributeTok{y =}\NormalTok{ smd, }\AttributeTok{v =}\NormalTok{ v\_smd))}
\NormalTok{\}}

\CommentTok{\# Example: Compute SMD for two groups}
\FunctionTok{compute\_SMD}\NormalTok{(}\AttributeTok{m1 =} \DecValTok{10}\NormalTok{, }\AttributeTok{m2 =} \DecValTok{8}\NormalTok{, }\AttributeTok{sd1 =} \DecValTok{2}\NormalTok{, }\AttributeTok{sd2 =} \FloatTok{1.5}\NormalTok{, }\AttributeTok{n1 =} \DecValTok{50}\NormalTok{, }\AttributeTok{n2 =} \DecValTok{50}\NormalTok{)}
\end{Highlighting}
\end{Shaded}

\begin{verbatim}
         y      v
1 1.131371 0.0464
\end{verbatim}

\begin{center}\rule{0.5\linewidth}{0.5pt}\end{center}

\subsection{\texorpdfstring{3. \textbf{Fixed-Effect
Model}}{3. Fixed-Effect Model}}\label{fixed-effect-model}

\subsubsection{Model Specification}\label{model-specification}

\textbf{Equation}: \[
y_i = \beta_F + e_i, \quad e_i \sim N(0, v_i)
\]

\textbf{SEM Representation}: - No latent variables.\\
- Fixed parameter: \(\beta_F\) (common effect).\\
- Known sampling variance \(v_i\) fixed via definition variables.

\textbf{R Code}:

\begin{Shaded}
\begin{Highlighting}[]
\NormalTok{data\_fixed }\OtherTok{\textless{}{-}} \FunctionTok{data.frame}\NormalTok{(}
  \AttributeTok{y =} \FunctionTok{c}\NormalTok{(}\FloatTok{0.5}\NormalTok{, }\FloatTok{0.7}\NormalTok{, }\FloatTok{0.3}\NormalTok{),   }\CommentTok{\# Effect sizes}
  \AttributeTok{v =} \FunctionTok{c}\NormalTok{(}\FloatTok{0.1}\NormalTok{, }\FloatTok{0.15}\NormalTok{, }\FloatTok{0.2}\NormalTok{)   }\CommentTok{\# Sampling variances}
\NormalTok{)}

\NormalTok{fixed\_model }\OtherTok{\textless{}{-}} \FunctionTok{meta}\NormalTok{(}\AttributeTok{y =}\NormalTok{ y, }\AttributeTok{v =}\NormalTok{ v, }\AttributeTok{data =}\NormalTok{ data\_fixed, }\AttributeTok{model.name =} \StringTok{"Fixed Effect"}\NormalTok{)}
\FunctionTok{summary}\NormalTok{(fixed\_model)}
\end{Highlighting}
\end{Shaded}

\begin{verbatim}

Call:
meta(y = y, v = v, data = data_fixed, model.name = "Fixed Effect")

95% confidence intervals: z statistic approximation (robust=FALSE)
Coefficients:
             Estimate  Std.Error     lbound     ubound z value Pr(>|z|)  
Intercept1 5.1538e-01 2.1472e-01 9.4549e-02 9.3622e-01  2.4003  0.01638 *
Tau2_1_1   1.0000e-10         NA         NA         NA      NA       NA  
---
Signif. codes:  0 '***' 0.001 '**' 0.01 '*' 0.05 '.' 0.1 ' ' 1

Q statistic on the homogeneity of effect sizes: 0.4615385
Degrees of freedom of the Q statistic: 2
P value of the Q statistic: 0.7939227

Heterogeneity indices (based on the estimated Tau2):
                             Estimate
Intercept1: I2 (Q statistic)        0

Number of studies (or clusters): 3
Number of observed statistics: 3
Number of estimated parameters: 2
Degrees of freedom: 1
-2 log likelihood: 0.1660267 
OpenMx status1: 5 ("0" or "1": The optimization is considered fine.
Other values may indicate problems.)
\end{verbatim}

\begin{verbatim}
Warning in print.summary.meta(x): OpenMx status1 is neither 0 or 1. You are advised to 'rerun' it again.
\end{verbatim}

\textbf{Interpretation}: - \texttt{Estimate} = \(\hat{\beta}_F\) (common
effect).\\
- \texttt{Std.Error} = standard error of \(\hat{\beta}_F\).

\begin{center}\rule{0.5\linewidth}{0.5pt}\end{center}

\subsection{\texorpdfstring{4. \textbf{Random-Effects
Model}}{4. Random-Effects Model}}\label{random-effects-model}

\subsubsection{Model Specification}\label{model-specification-1}

\textbf{Equation}: \[
y_i = \beta_R + u_i + e_i, \quad u_i \sim N(0, \tau^2), \quad e_i \sim N(0, v_i)
\]

\textbf{SEM Representation}: - \textbf{Latent variable}:
\(f_i \sim N(\beta_R, \tau^2)\) (true effect).\\
- \textbf{Observed variable}: \(y_i = f_i + e_i\), with
\(e_i \sim N(0, v_i)\).

\textbf{Key Metrics}: - \(\tau^2\): Between-study variance.\\
- \(I^2 = \frac{\tau^2}{\tau^2 + \tilde{v}}\): Proportion of total
variance due to heterogeneity.

\textbf{R Code}:

\begin{Shaded}
\begin{Highlighting}[]
\NormalTok{random\_model }\OtherTok{\textless{}{-}} \FunctionTok{meta}\NormalTok{(}\AttributeTok{y =}\NormalTok{ y, }\AttributeTok{v =}\NormalTok{ v, }\AttributeTok{data =}\NormalTok{ data\_fixed, }\AttributeTok{model.name =} \StringTok{"Random Effects"}\NormalTok{)}
\FunctionTok{summary}\NormalTok{(random\_model)}
\end{Highlighting}
\end{Shaded}

\begin{verbatim}

Call:
meta(y = y, v = v, data = data_fixed, model.name = "Random Effects")

95% confidence intervals: z statistic approximation (robust=FALSE)
Coefficients:
             Estimate  Std.Error     lbound     ubound z value Pr(>|z|)  
Intercept1 5.1538e-01 2.1472e-01 9.4549e-02 9.3622e-01  2.4003  0.01638 *
Tau2_1_1   1.0000e-10         NA         NA         NA      NA       NA  
---
Signif. codes:  0 '***' 0.001 '**' 0.01 '*' 0.05 '.' 0.1 ' ' 1

Q statistic on the homogeneity of effect sizes: 0.4615385
Degrees of freedom of the Q statistic: 2
P value of the Q statistic: 0.7939227

Heterogeneity indices (based on the estimated Tau2):
                             Estimate
Intercept1: I2 (Q statistic)        0

Number of studies (or clusters): 3
Number of observed statistics: 3
Number of estimated parameters: 2
Degrees of freedom: 1
-2 log likelihood: 0.1660267 
OpenMx status1: 5 ("0" or "1": The optimization is considered fine.
Other values may indicate problems.)
\end{verbatim}

\begin{verbatim}
Warning in print.summary.meta(x): OpenMx status1 is neither 0 or 1. You are advised to 'rerun' it again.
\end{verbatim}

\begin{Shaded}
\begin{Highlighting}[]
\CommentTok{\# Calculate I²}
\NormalTok{I2 }\OtherTok{\textless{}{-}}\NormalTok{ random\_model}\SpecialCharTok{$}\NormalTok{I2.values}
\CommentTok{\#cat("I² =", round(I2, 2))}
\end{Highlighting}
\end{Shaded}

\begin{center}\rule{0.5\linewidth}{0.5pt}\end{center}

\subsection{\texorpdfstring{5. \textbf{Mixed-Effects
Model}}{5. Mixed-Effects Model}}\label{mixed-effects-model}

\subsubsection{Model Specification}\label{model-specification-2}

\textbf{Equation}: \[
y_i = \beta_0 + \beta_1 x_i + u_i + e_i, \quad u_i \sim N(0, \tau^2)
\]

\textbf{SEM Representation}: - \textbf{Latent variable}:
\(f_i \sim N(\beta_0 + \beta_1 x_i, \tau^2)\).\\
- \textbf{Observed variable}: \(y_i = f_i + e_i\), with
\(e_i \sim N(0, v_i)\).

\textbf{Interpretation}: - \(\beta_1\): Change in effect size per unit
increase in \(x_i\).\\
-
\(R^2 = \frac{\tau^2_{\text{without } x} - \tau^2_{\text{with } x}}{\tau^2_{\text{without } x}}\):
Variance explained by \(x_i\).

\textbf{R Code}:

\begin{Shaded}
\begin{Highlighting}[]
\CommentTok{\# Add moderator}
\NormalTok{data\_mixed }\OtherTok{\textless{}{-}} \FunctionTok{data.frame}\NormalTok{(}
  \AttributeTok{y =} \FunctionTok{c}\NormalTok{(}\FloatTok{0.5}\NormalTok{, }\FloatTok{0.7}\NormalTok{, }\FloatTok{0.3}\NormalTok{),}
  \AttributeTok{v =} \FunctionTok{c}\NormalTok{(}\FloatTok{0.1}\NormalTok{, }\FloatTok{0.15}\NormalTok{, }\FloatTok{0.2}\NormalTok{),}
  \AttributeTok{year =} \FunctionTok{c}\NormalTok{(}\DecValTok{2010}\NormalTok{, }\DecValTok{2015}\NormalTok{, }\DecValTok{2020}\NormalTok{)  }\CommentTok{\# Moderator}
\NormalTok{)}

\NormalTok{mixed\_model }\OtherTok{\textless{}{-}} \FunctionTok{meta}\NormalTok{(}\AttributeTok{y =}\NormalTok{ y, }\AttributeTok{v =}\NormalTok{ v, }\AttributeTok{x =}\NormalTok{ year, }\AttributeTok{data =}\NormalTok{ data\_mixed, }\AttributeTok{model.name =} \StringTok{"Mixed Effects"}\NormalTok{)}
\FunctionTok{summary}\NormalTok{(mixed\_model)}
\end{Highlighting}
\end{Shaded}

\begin{verbatim}

Call:
meta(y = y, v = v, x = year, data = data_mixed, model.name = "Mixed Effects")

95% confidence intervals: z statistic approximation (robust=FALSE)
Coefficients:
              Estimate   Std.Error      lbound      ubound z value Pr(>|z|)
Intercept1  2.7367e+01  1.0769e+02 -1.8370e+02  2.3843e+02  0.2541   0.7994
Slope1_1   -1.3333e-02  5.3473e-02 -1.1814e-01  9.1472e-02 -0.2493   0.8031
Tau2_1_1    1.0000e-10          NA          NA          NA      NA       NA

Q statistic on the homogeneity of effect sizes: 0.4615385
Degrees of freedom of the Q statistic: 2
P value of the Q statistic: 0.7939227

Explained variances (R2):
                       y1
Tau2 (no predictor)     0
Tau2 (with predictors)  0
R2                      0

Number of studies (or clusters): 3
Number of observed statistics: 3
Number of estimated parameters: 3
Degrees of freedom: 0
-2 log likelihood: 0.1044882 
OpenMx status1: 5 ("0" or "1": The optimization is considered fine.
Other values may indicate problems.)
\end{verbatim}

\begin{verbatim}
Warning in print.summary.meta(x): OpenMx status1 is neither 0 or 1. You are advised to 'rerun' it again.
\end{verbatim}

\begin{Shaded}
\begin{Highlighting}[]
\CommentTok{\# Calculate R²}
\NormalTok{tau2\_without\_x }\OtherTok{\textless{}{-}} \FunctionTok{meta}\NormalTok{(}\AttributeTok{y =}\NormalTok{ y, }\AttributeTok{v =}\NormalTok{ v, }\AttributeTok{data =}\NormalTok{ data\_mixed)}\SpecialCharTok{$}\NormalTok{tau2}
\NormalTok{tau2\_with\_x }\OtherTok{\textless{}{-}}\NormalTok{ mixed\_model}\SpecialCharTok{$}\NormalTok{tau2}
\NormalTok{R2 }\OtherTok{\textless{}{-}}\NormalTok{ (tau2\_without\_x }\SpecialCharTok{{-}}\NormalTok{ tau2\_with\_x) }\SpecialCharTok{/}\NormalTok{ tau2\_without\_x}
\FunctionTok{cat}\NormalTok{(}\StringTok{"R² ="}\NormalTok{, }\FunctionTok{round}\NormalTok{(R2, }\DecValTok{2}\NormalTok{))}
\end{Highlighting}
\end{Shaded}

\begin{verbatim}
R² = 
\end{verbatim}

\begin{center}\rule{0.5\linewidth}{0.5pt}\end{center}

\subsection*{\texorpdfstring{6. \textbf{Conceptual Deep
Dive}}{6. Conceptual Deep Dive}}\label{conceptual-deep-dive}
\addcontentsline{toc}{subsection}{6. \textbf{Conceptual Deep Dive}}

\begin{tcolorbox}[enhanced jigsaw, toptitle=1mm, colframe=quarto-callout-tip-color-frame, left=2mm, colbacktitle=quarto-callout-tip-color!10!white, opacitybacktitle=0.6, bottomtitle=1mm, bottomrule=.15mm, coltitle=black, titlerule=0mm, title=\textcolor{quarto-callout-tip-color}{\faLightbulb}\hspace{0.5em}{Why SEM for Meta-Analysis?}, opacityback=0, toprule=.15mm, rightrule=.15mm, colback=white, breakable, arc=.35mm, leftrule=.75mm]

\begin{enumerate}
\def\labelenumi{\arabic{enumi}.}
\tightlist
\item
  \textbf{Latent Variables}:

  \begin{itemize}
  \tightlist
  \item
    Separate true effects (\(f_i\)) from sampling error (\(e_i\)).\\
  \item
    Example: If \(\tau^2 = 0\), all variability is due to sampling error
    (fixed-effect model).
  \end{itemize}
\item
  \textbf{Definition Variables}:

  \begin{itemize}
  \tightlist
  \item
    Fix known sampling variances (\(v_i\)) as constants per study.
  \end{itemize}
\item
  \textbf{Missing Data}:

  \begin{itemize}
  \tightlist
  \item
    FIML retains studies with incomplete data, unlike traditional
    listwise deletion.\\
  \end{itemize}
\end{enumerate}

\end{tcolorbox}

\begin{center}\rule{0.5\linewidth}{0.5pt}\end{center}

\subsection*{\texorpdfstring{7.
\textbf{Summary}}{7. Summary}}\label{summary}
\addcontentsline{toc}{subsection}{7. \textbf{Summary}}

\begin{longtable}[]{@{}
  >{\raggedright\arraybackslash}p{(\columnwidth - 6\tabcolsep) * \real{0.1852}}
  >{\raggedright\arraybackslash}p{(\columnwidth - 6\tabcolsep) * \real{0.3241}}
  >{\raggedright\arraybackslash}p{(\columnwidth - 6\tabcolsep) * \real{0.3241}}
  >{\raggedright\arraybackslash}p{(\columnwidth - 6\tabcolsep) * \real{0.1667}}@{}}
\toprule\noalign{}
\begin{minipage}[b]{\linewidth}\raggedright
Model
\end{minipage} & \begin{minipage}[b]{\linewidth}\raggedright
Equation
\end{minipage} & \begin{minipage}[b]{\linewidth}\raggedright
SEM Component
\end{minipage} & \begin{minipage}[b]{\linewidth}\raggedright
R Function
\end{minipage} \\
\midrule\noalign{}
\endhead
\bottomrule\noalign{}
\endlastfoot
Fixed-Effect & \(y_i = \beta_F + e_i\) & No latent variables &
\texttt{meta(y,\ v)} \\
Random-Effects & \(y_i = \beta_R + u_i + e_i\) & Latent
\(f_i \sim N(\beta_R, \tau^2)\) & \texttt{meta(y,\ v)} \\
Mixed-Effects & \(y_i = \beta_0 + \beta_1 x_i + u_i + e_i\) & Latent
\(f_i \sim N(\beta_0 + \beta_1 x_i, \tau^2)\) &
\texttt{meta(y,\ v,\ x)} \\
\end{longtable}

\begin{tcolorbox}[enhanced jigsaw, toptitle=1mm, colframe=quarto-callout-important-color-frame, left=2mm, colbacktitle=quarto-callout-important-color!10!white, opacitybacktitle=0.6, bottomtitle=1mm, bottomrule=.15mm, coltitle=black, titlerule=0mm, title=\textcolor{quarto-callout-important-color}{\faExclamation}\hspace{0.5em}{Advantages of SEM-Based Meta-Analysis}, opacityback=0, toprule=.15mm, rightrule=.15mm, colback=white, breakable, arc=.35mm, leftrule=.75mm]

\begin{itemize}
\tightlist
\item
  \textbf{Flexibility}: Extend to multivariate/multilevel models.\\
\item
  \textbf{Precision}: Directly model heterogeneity as latent variance.\\
\item
  \textbf{Robustness}: Integrate with SEM's estimation tools (e.g.,
  FIML, constraints).\\
\end{itemize}

\end{tcolorbox}




\end{document}
